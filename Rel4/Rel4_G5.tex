\documentclass[a4 paper]{article}
%\usepackage{minted}           %embedding code
\usepackage{amsmath, amsthm, amsfonts} %always use amsmath for symbols, amsthm for theorems 
\usepackage{graphicx}  % for pictures
%\usepackage{lipsum}  % for test text
\usepackage{multicol}    % for multicollumn text
\usepackage[bottom=2.5cm]{geometry}   %to set the margins to your liking
\usepackage[skip = 10pt, indent = 30pt]{parskip}      %to set the distance between paragraphs
\usepackage{tcolorbox}           %for literal color boxes
%\usepackage{witharrows}             % understandable, arrows for equations
\usepackage{tikz}                   %drawings and diagrams
\usetikzlibrary{positioning}        %tikz library for positioning (of nodes?)
\usepackage{pgfplots}               %plotting and graphs
\pgfplotsset{compat=1.18, width = 10cm}
\usepackage{hyperref}
\hypersetup{colorlinks = true, linkcolor = black, urlcolor = blue}
%\usepackage{fancyvrb}           % fancy formatting of verbatim
%\usepackage{fancyhdr, lastpage}
%\pagestyle{fancy} 
%\lhead{Relat\'orio experimento 4}
%\rhead{FisExpI}
%\cfoot{Página \thepage \ de \pageref{LastPage}}
%\usepackage[Bjornstrup]{fncychap} %Sonny, Glenn, Lenny, Conny, Rejne, Bjarne, Bjornstrup
%\usepackage{xcolor}      %color text
\usepackage{siunitx}    %for SI units
\usepackage{setspace}
\onehalfspacing
\usepackage{cleveref}
\usepackage[brazil]{babel}
\usepackage{caption}
\usepackage{subcaption}
\usepackage{pdfpages}
\usepackage{booktabs}
\usepackage{multirow}
\usepackage{textcomp}
\usepackage{amssymb}
\usepackage[document]{ragged2e}
\usepackage{bm}
\usepackage{empheq}




%\setlength{\hoffset}{-2cm}
%\setlength{\voffset}{1.5cm}                     %control your margins however you want!
%\setlength{\marginparwidth}{2cm}
%\setlength{\oddsidemargin}{0cm}

%\newtheorem{theorem}{Theorem}[section]               %how you call it and how you display it
%\newtheorem{corollary}{Corollary}[theorem]


\newcommand{\parag}{\hspace{30pt}}
%\newcommand{\pd}[2]{\frac{\partial#1}{\partial#2}}


\begin{document}
\justifying
\begin{center}{\large Laboratório de Circuitos Elétricos - 02/2024 - Turma 05}\\
{\large \textbf{Experimento 4}}\\ 
28/11/2024
\end{center}

\vspace{500pt}
 \noindent\textbf{Grupo 5:}\\
 Yuri Shumyatsky - 231012826\\
Vinicius de Melo Moraes - 231036274\\
Victor Rizzi Wagner - 231012817


\vspace{30pt}
\newpage

\section{Introdução}

\parag Serão analisados circuitos elétricos de primeira e segunda ordem em experimentos realizados com alimentação por uma fonte de corrente alternada (AC), para que seja possível observar o efeito da mudança de tensão continuamente, em vez de em apenas um instante. Circuitos de primeira ordem, compostos por resistores e capacitores (RC) ou resistores e indutores (RL), apresentam respostas dinâmicas caracterizadas por uma única constante de tempo, enquanto circuitos de segunda ordem, como os RLC, possuem respostas mais complexas, que podem ser oscilatórias ou amortecidas, dependendo de seus parâmetros.

O objetivo do experimento foi investigar o comportamento desses circuitos quando submetidos a uma mudança brusca de tensão, analisando aspectos como amplitude, fase e frequência das grandezas elétricas envolvidas. Através da montagem prática dos circuitos e da medição das tensões e correntes em diferentes componentes, buscou-se validar os modelos teóricos e compreender os fenômenos de ressonância, amortecimento e mudanças de fase.

\vspace{70pt}
\section{Materiais}

	\begin{itemize}
	\item National Instruments Elvis II
	\item 1 capacitor de 47n$C$
	\item 1 indutor de 1m$H$
	\item 1 resistor de 1k$\Omega$
	\item 1 resistor de 47$\Omega$
	\end{itemize}

\newpage
\section{Procedimento}

\parag O National Instruments Elvis é usado como fonte, protoboard, e multímetro. Usa-se a função de multímetro para checar as resistências, capacitância e indutância dos componentes, que são marcadas na Tabela 1.

\vspace{5pt}
\begin{table}[h]
\centering
\begin{tabular}{|c|c|c|c|}
\hline
Grandeza & Valor nominal & Valor medido & Erro (\%) \\\hline
C & 47nF & 46,58nF & \\    \hline
L & 1mH & 0,8694mH & \\    \hline
$R_1$ & 1k$\Omega$ & 0,986k$\Omega$ & \\\hline
$R_2$ & 47$\Omega$ & 46,424$\Omega$ & \\\hline
\end{tabular}
\caption*{Tabela 1: Componentes}
\end{table}

Em seguida, é montado o circuito da Figura 1, usando $R_1=1k\Omega$.

\begin{table}[h]
\centering
\includegraphics[scale=0.3]{figuras/figura1}
\end{table}

\begin{center}
Figura 1: Circuito de primeira ordem
\end{center}

\vspace{5pt}
\begin{table}[h]
\centering
\begin{tabular}{|c|c|c|c|}
\hline
Tensão & Valor nominal (V) & Valor medido (V) & Erro (\%) \\\hline
$V_1(0)$ &  & -976,53mV & \\    \hline
$V_1(\tau)$ &  & 281,17mV & \\    \hline
$V_1(2\tau)$ &  & 700,40mV & \\\hline
$V_1(3\tau)$ &  & 910,00mV & \\\hline
$V_1(10\tau)$ &  & 1,04V & \\\hline
\end{tabular}
\caption*{Tabela 2: Tensões para circuito RC}
\end{table}

\vspace{5pt}
\begin{table}[h]
\centering
\begin{tabular}{|c|c|c|c|}
\hline
Tensão & Valor nominal (V) & Valor medido (V) & Erro (\%) \\\hline
$V_1(0)$ &  & -976,53mV & \\    \hline
$V_1(\tau_1)$ &  & 197,32mV & \\    \hline
$V_1(2\tau_1)$ &  & 700,40mV & \\\hline
$V_1(3\tau_1)$ &  & 910,02mV & \\\hline
$V_1(10\tau_1)$ &  & 1,04V & \\\hline
\end{tabular}
\caption*{Tabela 3: Tensões para circuito RLC}
\end{table}

\vspace{5pt}
\begin{table}[h]
\centering
\begin{tabular}{|p{5cm}|c|c|c|}
\hline
Grandeza & Valor nominal & Valor medido & Erro (\%) \\\hline
Tempo para $V_1$ atingir seu valor máximo a partir de uma borda de subida da onda quadrada &  & 23,20$\mu$s & \\    \hline
\centering Valor máximo de $V_1$ &  & 1,41V & \\    \hline
\end{tabular}
\caption*{Tabela 4: Circuito RLC com resistência menor}
\end{table}

\newpage
\section{Conclusão}

\section{Bibliografia}
\begin{itemize}
\item HALLIDAY, D.; RESNICK, R.; WALKER, J. Fundamentos de Física. 10. ed. v. 3. Rio de Janeiro: LTC, 2016.
\end{itemize}


\end{document}